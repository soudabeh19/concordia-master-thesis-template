\chapter{Conclusion}
% berif introduction of this chapter and what we are going to cover in this chapter 
% why we choosed this dataset (importance of reproducibility in neuroscience, huge files with long processing time)  
\section{challenges}
\begin{tcolorbox}
        \begin{itemize}
                \item brief reviw of our challenges in this study (file generation order)
                \item the more differences at the first of pipeline the better results we get
                (this means our methods are dependent to the pattern of utility metrices but in general they are applicable on PreFreesurfer pipline)
                \item if we do the experiment on the other pipelines we can mention that process
                of subjects are time consuming and we had to complete one pipeline to start working on the other one and do the experiment
                \item take CentOS6 vs CentOS7 and present the superimposed-plot "all" methods with training ratio of 0.6
\end{itemize}
\end{tcolorbox}
\section{Applications}
\begin{tcolorbox}
        \begin{itemize}
                \item if we do experiment on decimal numeral range and get sure about our methods
                performance we can say our method can predict the time generation file of the subject
                so we can handle when to stop the pipeline or when to resume it.
                \item use to predict the content of files by use of their binary code
                (or atleast we can antisipate files with the same content in all subject.
                But the question is when we have to respect the file generation order how we can take the advantage of knowing constant files)
\end{itemize}
\end{tcolorbox}
\section{Future works}
\begin{tcolorbox}
        \begin{itemize}
                \item checking by appending the utility matrix of two follow up pipelines (Freesurfer,PostFreesurfer)
                to the PreFreesurfer utility matrix if we could be able to improve our prediction results over the results
                of just Bias
                \item the correlation of random selected subjects and accuracy (in this case by classifing subjects maybe we could get better result)

\end{itemize}
\end{tcolorbox}
